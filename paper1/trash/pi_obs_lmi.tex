To formulate the $\mathbb{H}_\infty$ problem for the PI observer with $z=C_z e$, let $P_1 \in \mathbb{S}_{++}$, $P_2 \in \mathbb{S}_{++}^p$
and $P_{12} \in \mathbb{R}^{n\times p}$, the Lyapunov matrix is chosen as,
\begin{gather}
P =\begin{bmatrix}
P_1 & P_{12} \\
P_{12}^\top & P_2
\end{bmatrix} > 0
\end{gather}
The corresponding LMI problem is given as,
\begin{gather}
\begin{split} 
\min_Y{(\gamma)}\quad \mathrm{s.t.}\\
\begin{bmatrix}
\Psi_{11} & \Psi_{12} & \Psi_{13} & C_z^\top \\
\star     & \Psi_{22} & \Psi_{23} & 0 \\
\star     & \star     & -\gamma I & 0 \\
\star     & \star     & \star     & -\gamma I
\end{bmatrix}\prec 0, \,
\begin{bmatrix}
P_1 & P_{12} \\
P_{12}^\top & P_2
\end{bmatrix}\succ 0
\end{split} 
\end{gather}
and 
\begin{gather}
\begin{split} 
\Psi_{11}&=
P_1 A + A^\top P_1
- Y C - C^\top Y^\top
+ P_{12} C + C^\top P_{12}^\top ,
\\[1mm]
\Psi_{12}
&=
A^\top P_{12} - Y_i ,
\\[1mm]
\Psi_{22}
&=
P_{12}^\top C + C^\top P_{12} ,
\\[1mm]
\Psi_{13}
&=
P_1 B_w - Y ,
\\[1mm]
\Psi_{23}
&=
P_{12}^\top B_w .
\end{split} 
\end{gather}
where $L = P_1^{-1} Y$ and $L_i = P_1^{-1} Y_i$.



The P Observer utilizes a proportional term to correct its estimate which results in steady-state error in its estimate when exogenous input is present. The Proportional-Integral Observer(PIO) is defined as follows,
\begin{gather} 
\begin{split} 
    \dot{\hat{x}}=A \hat{x}+B_u u+L(y-\hat{y})+L_i q\\
    \dot{q}=y-\hat{y},\,\hat{y}=C \hat{x}
\end{split} 
\end{gather}
which is used to compansate for steady state errors in the estimate for constant exogenous inputs. The corresponding error dynamics are obtained as,
\begin{gather} 
    \begin{bmatrix}
        \dot{e}\\\dot{q}
    \end{bmatrix}=\begin{bmatrix}
        A-LC& -L_i\\C& 0
    \end{bmatrix}\begin{bmatrix}
        e\\q
    \end{bmatrix}+\begin{bmatrix}
        B_w-D_wL\\D_w
    \end{bmatrix}w
\end{gather} 
The integral term $L_i$ is chosen for fixed $L$ for the sake of simplicity.


%%%%%%%%%%%%%%%%%%%%%%%%%%%%%%%%%%%%%%%%%%%%%%%%%%%%%%%%%%%%%%%%%%%%%%%%%%%%%%%%%%%%%%%%%%%%%%%%%%%%%%%%%%%%%%%%%%%%%%%%%
\subsection{The GI-PI Observer}
%%%%%%%%%%%%%%%%%%%%%%%%%%%%%%%%%%%%%%%%%%%%%%%%%%%%%%%%%%%%%%%%%%%%%%%%%%%%%%%%%%%%%%%%%%%%%%%%%%%%%%%%%%%%%%%%%%%%%%%%%
The proposed GI-PI observer is defined as,
\begin{gather} 
    \dot{\hat{x}}=A \hat{x}+B_u u+L(y-\hat{y})+L_i q+B_w \Pi y\\\nonumber
    \dot{q}=y-\hat{y}\\
    \hat{y}=C \hat{x}+\Pi y\nonumber
\end{gather}
with error dynamics,
\begin{gather}
    \begin{bmatrix}
        \dot{e}\\\dot{q}
    \end{bmatrix}
    =\begin{bmatrix}
        (A-LC)& (B_w-L)C\\
        -L_i& 0
    \end{bmatrix}\begin{bmatrix}
        e\\q
    \end{bmatrix}
\end{gather}