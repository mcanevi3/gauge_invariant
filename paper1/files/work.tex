\section{Work}
%%%%%%%%%%%%%%%%%%%%%%%%%%%%%%%%%%%%%%%%%%%%%%%%%%%%%%%%%%%%%%%%%%%%%%%%%%%%%%%%%%%%%%%%%%%%%%%%%%%%%%%%%%%%%%%%%%%%%%%%%
\subsection{The System Model}
%%%%%%%%%%%%%%%%%%%%%%%%%%%%%%%%%%%%%%%%%%%%%%%%%%%%%%%%%%%%%%%%%%%%%%%%%%%%%%%%%%%%%%%%%%%%%%%%%%%%%%%%%%%%%%%%%%%%%%%%%
Let $x\in\mathbb{R}^{n\times1}$, $y\in\mathbb{R}^{p\times1}$, $w\in\mathbb{R}^{r\times1}$ be the state, measurement, exogenous vectors, respectively, the SIMO/MIMO LTI system addressed in this paper is stated as, 
\begin{gather} 
    \dot{x}=A x+B_w w+B_u u,\,y=C x+D_w w
\end{gather}
where $A\in\mathbb{R}^{n\times n}$, $B_w\in\mathbb{R}^{n\times r}$, $C\in\mathbb{R}^{p\times n}$ and $D_w\in\mathbb{R}^{p\times r}$. The following additional rank condition
\begin{gather}
    \mathrm{rank}(C)=r<p
\end{gather}
arises in sparse sensor applications\cite{shoukry2015event}, topologies used in Multi Agen Systems(MAS)\cite{olfati2007consensus} and distributed networks \cite{yang2022sensor}, and is also called Strictly Output Redundant(SOR) system \cite{yang2025output}. Here, the system output is overdetermined, therefore, 
\begin{gather}  
    \mathrm{dim}\,\mathcal{N}(C^T)=p-r\geq 1
\end{gather}  
or using orthogonality $C^TC^\perp=0$,
\begin{gather}  
    \mathrm{dim}\,\mathcal{N}(C^\perp)=p-r\geq 1
\end{gather}  
which ensures nontrivial orthogonal component in the output space. 

The system model is augmented with constant exogenous inputs as follows,
\begin{gather} \label{eq:augmented_system}
    \begin{bmatrix}\dot{x}\\\dot{w}\end{bmatrix}=\begin{bmatrix}
        A& B_w\\0& 0
    \end{bmatrix}
    \begin{bmatrix}
        x\\w
    \end{bmatrix}
    +\begin{bmatrix}
        B_u\\0
    \end{bmatrix}u,\,
    y=\begin{bmatrix}
       C& D_w
    \end{bmatrix}
    \begin{bmatrix}
        x\\w
    \end{bmatrix}
\end{gather}
and in short form as,
\begin{gather} 
\dot{\tilde{x}}=\tilde{A}\tilde{x}+\tilde{B}u,\,y=\tilde{C}\tilde{x}
\end{gather}
where $\tilde{x}\triangleq \begin{bmatrix}
        x&w
    \end{bmatrix}^T$ and matrices with tilde are as in Eq~\ref{eq:augmented_system}.

%%%%%%%%%%%%%%%%%%%%%%%%%%%%%%%%%%%%%%%%%%%%%%%%%%%%%%%%%%%%%%%%%%%%%%%%%%%%%%%%%%%%%%%%%%%%%%%%%%%%%%%%%%%%%%%%%%%%%%%%%
\subsection{The Observer Models}
%%%%%%%%%%%%%%%%%%%%%%%%%%%%%%%%%%%%%%%%%%%%%%%%%%%%%%%%%%%%%%%%%%%%%%%%%%%%%%%%%%%%%%%%%%%%%%%%%%%%%%%%%%%%%%%%%%%%%%%%%
The classical Luenberger Observer, in this context also called P Observer(PO) is defined as,
\begin{gather} 
    \dot{\hat{x}}=A \hat{x}+B_u u+L(y-\hat{y}),\,\hat{y}=C \hat{x}
\end{gather}
Assuming $(A,C)$-pair observable, the error dynamics are obtained for the error $e\triangleq x-\hat{x}$ as,
\begin{gather} 
    \dot{e}=(A-LC)e+(B_w-D_wL)w
\end{gather}
The $\mathbb{H}_\infty$ optimal observer for objective function $z=C_ze$ is designed using the following LMI problem\cite{duan2013lmis},
\begin{equation*} \label{eq:p_lmi1}
\begin{split} 
    \min_Y{(\gamma)}\quad \mathrm{s.t.}\\
    \begin{bmatrix}
        (PA-YC)+(PA-YC)^T& PB_w-YD_w& C_z^T\\
        \star& -\gamma I& 0\\
        \star& 0& -\gamma I
    \end{bmatrix}\prec 0\\
    P\succ 0
\end{split} 
\end{equation*}
where the observer gain is recovered with $L=P^{-1}Y$. For constant exogenous inputs the augmented LO is given as,
\begin{gather} 
    \begin{bmatrix}\dot{\hat{x}}\\\dot{\hat{w}}\end{bmatrix}=\begin{bmatrix}
        A& B_w\\0& 0
    \end{bmatrix}
    \begin{bmatrix}
        \hat{x}\\\hat{w}
    \end{bmatrix}+\begin{bmatrix}
        L_1\\L_2
    \end{bmatrix}(y-\hat{y})
    +\begin{bmatrix}
        B_u\\0
    \end{bmatrix}u\\
    \hat{y}=\begin{bmatrix}
       C& D_w
    \end{bmatrix}
    \begin{bmatrix}
        \hat{x}\\\hat{w}
    \end{bmatrix}
\end{gather}
with error dynamics,
\begin{gather}
    \begin{bmatrix}
        \dot{e}\\\dot{e}_w
    \end{bmatrix}
    =\begin{bmatrix}
        A-L_1C& B_w-L_1D_w\\
        -L_2C& -L_2
    \end{bmatrix}
    \begin{bmatrix}
        e\\e_w
    \end{bmatrix}+\begin{bmatrix}
        B_u\\0
    \end{bmatrix}u
\end{gather}

%%%%%%%%%%%%%%%%%%%%%%%%%%%%%%%%%%%%%%%%%%%%%%%%%%%%%%%%%%%%%%%%%%%%%%%%%%%%%%%%%%%%%%%%%%%%%%%%%%%%%%%%%%%%%%%%%%%%%%%%%
\subsection{The Projection}
%%%%%%%%%%%%%%%%%%%%%%%%%%%%%%%%%%%%%%%%%%%%%%%%%%%%%%%%%%%%%%%%%%%%%%%%%%%%%%%%%%%%%%%%%%%%%%%%%%%%%%%%%%%%%%%%%%%%%%%%%
The projection defined using Gauge-Invariance principle is "proposed" as follows,
\begin{gather}
    \Pi\triangleq I-C(C^TC)^{-1}C^T
\end{gather}
where the following properties are satisfied,
\begin{gather}
\begin{split}
    &\Pi C= (I-C(C^TC)^{-1}C^T)C=0\\
    &\Pi C^\perp=(I-C(C^TC)^{-1}C^T)C^\perp=C^\perp,\,C^TC^\perp=0\\
    &\Pi^2=\Pi
\end{split}
\end{gather}
Using the projection on the measured output gives,
\begin{gather}
    \Pi y=\Pi(Cx+D_w w)=\Pi D_w w=\Pi \tilde{w}
\end{gather}
The projection eliminates the components in the output space and the orthogonal components remain. The exogenous signal $\tilde{w}$ is defined to deal with the $D_w$ matrix. The exogenous signal is decomposed as follows,
\begin{gather}
    \tilde{w}=C\alpha+C^\perp \beta
\end{gather}
After this decomposition the projection on the output gives,
\begin{gather}
    \Pi y=\Pi(Cx+C\alpha+C^\perp \beta)=\Pi C^\perp \beta=C^\perp \beta
\end{gather}
which is the orthogonal component. 

%%%%%%%%%%%%%%%%%%%%%%%%%%%%%%%%%%%%%%%%%%%%%%%%%%%%%%%%%%%%%%%%%%%%%%%%%%%%%%%%%%%%%%%%%%%%%%%%%%%%%%%%%%%%%%%%%%%%%%%%%
\subsection{The GI-Luenberger Observer}
%%%%%%%%%%%%%%%%%%%%%%%%%%%%%%%%%%%%%%%%%%%%%%%%%%%%%%%%%%%%%%%%%%%%%%%%%%%%%%%%%%%%%%%%%%%%%%%%%%%%%%%%%%%%%%%%%%%%%%%%%
The classical Luenberger Observer is fed with this orthogonal component forming the GI-Luenberger observer, the observer expression becomes,
\begin{gather} 
    \dot{\hat{x}}=A \hat{x}+B_u u+L(y-\hat{y})+B_w \Pi y,\,\hat{y}=C \hat{x}+\Pi y
\end{gather}
The error dynamics of LO,
\begin{gather}
    \dot{e}=(A-LC)e+(B_w-L)C\alpha+(B_w-L)C^\perp\beta
\end{gather}
is as follows for the GI-Luenberger observer,
\begin{gather}
    \dot{e}=(A-LC)e+(B_w-L)C\alpha
\end{gather}
Assuming the observer is designed using the LMI given in Eq~\ref{eq:p_lmi1}, the main difference is in the exogenous input decomposition, thus the performance depends on the decomposition. The steady-state estimation errors will persist for both the LO and GI-LO even for constant exogenous inputs. 
Assuming constant exogenous inputs, the following GI-Observer is stated,
\begin{gather} 
    \begin{bmatrix}\dot{\hat{x}}\\\dot{\alpha}\end{bmatrix}=
        \begin{bmatrix}A& B_w\\0& 0\end{bmatrix}
        \begin{bmatrix}\hat{x}\\\alpha\end{bmatrix}
        +L(y-\hat{y})+\begin{bmatrix}B_w\\0\end{bmatrix} \Pi y,\\
        \hat{y}=\begin{bmatrix}C& I\end{bmatrix} \begin{bmatrix}\hat{x}\\\alpha\end{bmatrix}+\Pi y
\end{gather}
with error dynamics,
\begin{gather} 
    \begin{bmatrix}\dot{e}_{x}\\\dot{e}_\alpha\end{bmatrix}=
        \left(\begin{bmatrix}A& B_w\\0& 0\end{bmatrix}-
        \begin{bmatrix}L_1\\L_2\end{bmatrix}\begin{bmatrix}C&I\end{bmatrix}
        \right)
        \begin{bmatrix}e_x\\e_\alpha\end{bmatrix}
\end{gather}


