\section{Work}
%%%%%%%%%%%%%%%%%%%%%%%%%%%%%%%%%%%%%%%%%%%%%%%%%%%%%%%%%%%%%%%%%
\subsection{The System Model}
%%%%%%%%%%%%%%%%%%%%%%%%%%%%%%%%%%%%%%%%%%%%%%%%%%%%%%%%%%%%%%%%%
Let $x\in\mathbb{R}^{n\times 1}$, $y\in\mathbb{R}^{p\times 1}$ and $w\in\mathbb{R}^{r\times 1}$ be the state, measurement, exogenous vectors, respectively, the SIMO/MIMO LTI system addressed in this paper is stated as, 
\begin{gather} 
    \dot{x}=A x+B_w w+B_u u,\,y=C x+D_w w
\end{gather}
where $A\in\mathbb{R}^{n\times n}$, $B_w\in\mathbb{R}^{n\times r}$, $C\in\mathbb{R}^{p\times n}$ and $D_w\in\mathbb{R}^{p\times r}$. The following additional rank condition
\begin{gather}
    \mathrm{rank}(C)=r<p
\end{gather}
arises in sparse sensor applications\cite{shoukry2015event}, topologies used in Multi Agen Systems(MAS)\cite{olfati2007consensus} and distributed networks \cite{yang2022sensor}, and is also called Strictly Output Redundant(SOR) system \cite{yang2025output}. Here, the system output is overdetermined, therefore, 
\begin{gather}  
    \mathrm{dim}\,\mathcal{N}(C^T)=p-r\geq 1
\end{gather}  
or using orthogonality $C^TC^\perp=0$,
\begin{gather}  
    \mathrm{dim}\,\mathcal{N}(C^\perp)=p-r\geq 1
\end{gather}  
which ensures nontrivial orthogonal component in the output space. 
%%%%%%%%%%%%%%%%%%%%%%%%%%%%%%%%%%%%%%%%%%%%%%%%%%%%%%%%%%%%%%%%%
\subsection{The Observer Models}
%%%%%%%%%%%%%%%%%%%%%%%%%%%%%%%%%%%%%%%%%%%%%%%%%%%%%%%%%%%%%%%%%
The classical Luenberger Observer is defined as,
\begin{gather} 
    \dot{\hat{x}}=A \hat{x}+B_u u+L(y-\hat{y}),\,\hat{y}=C \hat{x}
\end{gather}
where $L\in\mathbb{R}^{n\times p}$ is the observer gain. Assuming $(A,C)$-pair observable, the error dynamics are obtained for the error $e\triangleq x-\hat{x}$ as,
\begin{gather} 
    \dot{e}=(A-LC)e+(B_w-L D_w)w
\end{gather}
The $\mathbb{H}_\infty$ optimal observer for objective function $z=C_ze$ is designed using the following LMI problem\cite{duan2013lmis},
\begin{equation*} \label{eq:p_lmi1}
\begin{split} 
    \min_{Y,P}{(\gamma)}\quad \mathrm{s.t.}\\
    \begin{bmatrix}
        (PA-YC)+\star^T& PB_w-YD_w& C_z^T\\
        \star& -\gamma I& 0\\
        \star& 0& -\gamma I
    \end{bmatrix}\prec 0\\
    P\succ 0
\end{split} 
\end{equation*}
where the observer gain is recovered with $L=P^{-1}Y$. 

%%%%%%%%%%%%%%%%%%%%%%%%%%%%%%%%%%%%%%%%%%%%%%%%%%%%%%%%%%%%%%%%%
\subsection{The Projection}
%%%%%%%%%%%%%%%%%%%%%%%%%%%%%%%%%%%%%%%%%%%%%%%%%%%%%%%%%%%%%%%%%
The projection operator defined using Gauge-Invariance principle is "proposed" as follows,
\begin{gather}
    \Pi\triangleq I-C(C^TC)^{-1}C^T
\end{gather}
which satisfies the following properties,
\begin{gather}
\begin{split}
    &\Pi C= (I-C(C^TC)^{-1}C^T)C=0\\
    &\Pi C^\perp=(I-C(C^TC)^{-1}C^T)C^\perp=C^\perp,\,C^TC^\perp=0\\
    &\Pi^2=\Pi,\,\Pi\in\mathbb{R}^{p\times p}
\end{split}
\end{gather}
Using the projection operator on the measured output yields,
\begin{gather}
    \Pi y=\Pi(C x+D_w w)=\Pi D_w w
\end{gather}
The projection eliminates the state dependent component of the output, isolating the exogenous component establishing state invariance. This is an algebraic measurement of the disturbance reflected to the output.The exogenous input is recovered under the following condition,
\begin{gather}
    \mathrm{rank}(\Pi D_w)=r
\end{gather}
or equivalently,
\begin{gather}
\ker(\Pi D_w) = {0}.
\end{gather}
with the least square estimator,
\begin{gather}
    \hat{\omega}=(\Pi D_w)^\dagger(\Pi y)
\end{gather}
where $(.)^\dagger$ is the Moore-Penrose pseudoinverse. The exogenous signal recovery fails if 
\begin{gather}
D_w w \in \mathrm{Im}(C),
\end{gather}
otherwise,
\begin{gather}
    \hat{\omega}\rightarrow \omega
\end{gather}
%%%%%%%%%%%%%%%%%%%%%%%%%%%%%%%%%%%%%%%%%%%%%%%%%%%%%%%%%%%%%%%%%
\subsection{The GI-Luenberger Observer}
%%%%%%%%%%%%%%%%%%%%%%%%%%%%%%%%%%%%%%%%%%%%%%%%%%%%%%%%%%%%%%%%%
The classical Luenberger observer is fed with the sensed exogenous input forming the GI-Luenberger observer, the observer expression becomes,
\begin{gather} 
\begin{split} 
    \dot{\hat{x}}=A \hat{x}+B_w\hat{\omega}+B_u u+L(y-\hat{y})\\
    \hat{y}=C \hat{x}+D_w \hat{\omega}\\
    \hat{\omega}=(\Pi D_w)^\dagger(\Pi y)
\end{split}
\end{gather}
the error dynamics are obtained as,
\begin{gather}
    \dot{e}=(A-LC)e
\end{gather}

The exogenous input is recovered since,
\begin{gather}
    \Pi y=\Pi C x+\Pi D_w w=\Pi D_w w
\end{gather}
and,
\begin{gather}
    (\Pi D_w)^\dagger \Pi y=(\Pi D_w)^\dagger (\Pi D_w) w=w
\end{gather}
under the following condition, where $D_w w$ is partitioned as,
\begin{gather}
    D_w=C\alpha+C^\perp \beta,\,\beta\neq 0
\end{gather}
since,
\begin{gather}
    \Pi D_w=\Pi C\alpha+\Pi C^\perp \beta=\Pi C^\perp \beta,\,\beta\neq 0
\end{gather}
and finally,
\begin{gather}
    \Pi D_w\neq 0
\end{gather}


%%%%%%%%%%%%%%%%%%%%%%%%%%%%%%%%%%%%%%%%%%%%%%%%%%%%%%%%%%%%%%%%%
\subsection{Robust Observer}
%%%%%%%%%%%%%%%%%%%%%%%%%%%%%%%%%%%%%%%%%%%%%%%%%%%%%%%%%%%%%%%%%
Let the system be given as,
\begin{gather} 
    \dot{x}=A x+B_w w,\,y=(C_0+\Delta C_0) x+D_w w
\end{gather}
where $||\Delta||_\infty\leq 1$, the projection,
\begin{gather}
    \Pi\triangleq I-C_0(C_0^TC_0)^{-1}C_0^T
\end{gather}
on the output gives,
\begin{gather}
    \Pi(C_0 x+\Delta C_0 x+D_w w)=\Pi \Delta C_0 x+\Pi D_w w
\end{gather}
The uncertainty model,
\begin{gather}
    \Delta C_0=H \Delta E
\end{gather}
where $\Delta^T\Delta\leq I$ or $||\Delta||_\infty\leq 1$. Hence,
\begin{gather}
    \Pi y=\Pi H \Delta E x+\Pi D_w w
\end{gather}
For 
\begin{gather}
    C_0=\begin{bmatrix} 
        2\\1
    \end{bmatrix}
\end{gather}
Case 1: $H=C_0$, $\Delta=\delta$, and $E=1$ hence,
\begin{gather}
    C=C_0+C_0\delta=C_0(1+\delta)
\end{gather}
Case 2: 
\begin{gather}
    H=\begin{bmatrix} 
        1&0\\0&2\\
    \end{bmatrix}
\end{gather}
and
\begin{gather}
    \Delta=\begin{bmatrix} 
        \delta_1\\\delta_2\\
    \end{bmatrix}
\end{gather}
and $E=1$ hence,
\begin{gather}
    C=C_0+
    \begin{bmatrix} 
        2\\0
    \end{bmatrix}\delta_1+
    \begin{bmatrix} 
        0\\1
    \end{bmatrix}\delta_2
\end{gather}
where $\delta_1\neq \delta_2$.

\subsection{Recovery of $w$ and $\Delta$ with projection}
The system output with uncertainty is given as,
\begin{gather} 
    y=(C_0+H\Delta E) x+D_w w
\end{gather}
where the projection 
\begin{gather}
    \Pi\triangleq I-C_0(C_0^TC_0)^{-1}C_0^T
\end{gather}
is applied to the output, hence,
\begin{gather} 
\begin{split} 
    \Pi y&=\Pi C_0+\Pi H\Delta E x+\Pi D_w w\\
    \Pi y&=\Pi H\Delta E x+\Pi D_w w
\end{split} 
\end{gather}
is obtained. The following is obtained to factor out $\Delta  Ex$,
\begin{gather} \label{eqn:component_delta_x}
\begin{split} 
    (\Pi H )^\dagger\Pi y&=(\Pi H )^\dagger\Pi H\Delta Ex+(\Pi H )^\dagger\Pi D_w w\\
    H^\dagger \Pi^\dagger\Pi y&= \Delta E x+(\Pi H )^\dagger\Pi D_w w\\
    H^\dagger \Pi y&= \Delta Ex+H ^\dagger\Pi^\dagger\Pi D_w w\\
    H^\dagger \Pi y&= \Delta Ex+H ^\dagger\Pi D_w w
\end{split} 
\end{gather}
and to factor out $w$,
\begin{gather} 
\begin{split} \label{eqn:component_w}
    (\Pi D_w )^\dagger\Pi y&=(\Pi H )^\dagger\Pi H\Delta E x+(\Pi H )^\dagger\Pi D_w w\\
    D_w^\dagger \Pi^\dagger\Pi y &=(\Pi D_w )^\dagger\Pi H \Delta Ex+(\Pi D_w )^\dagger\Pi D_w w\\
    D_w^\dagger \Pi y&=(\Pi D_w )^\dagger\Pi H \Delta Ex+ w\\
    &=D_w^\dagger \Pi^\dagger\Pi H \Delta Ex+ w\\
    D_w^\dagger \Pi y&=D_w^\dagger \Pi H \Delta E x+ w\\
\end{split} 
\end{gather}
Both Eq~\ref{eqn:component_delta_x} and Eq~\ref{eqn:component_w} are combined into,
\begin{gather} 
\begin{split} 
    \begin{bmatrix}
        I & H ^\dagger\Pi D_w\\
        D_w^\dagger \Pi H& I
    \end{bmatrix}
    \begin{bmatrix}
       \Delta E x\\w
    \end{bmatrix}=
    \begin{bmatrix}
       H^\dagger\\D_w^\dagger
    \end{bmatrix}\Pi y
\end{split} 
\end{gather}
The singularity,
\begin{gather} 
\begin{split} 
    \left|\begin{array}{cc}
        I & H ^\dagger\Pi D_w\\
        D_w^\dagger \Pi H & I
    \end{array}\right|=I-H ^\dagger\Pi D_w D_w^\dagger \Pi H
\end{split} 
\end{gather}
is obtained. $w$ is recovered with,
\begin{gather} 
\begin{split} 
    w&=\frac{\left|\begin{array}{cc}
        I & H^\dagger\\
        D_w^\dagger \Pi H & D_w^\dagger
    \end{array}\right|}
    {
    \left|\begin{array}{cc}
        I & H ^\dagger\Pi D_w\\
        D_w^\dagger \Pi H & I
    \end{array}\right|
    }\\
    &=\frac{D_w^\dagger-H^\dagger D_w^\dagger \Pi H}{I-H ^\dagger\Pi D_w D_w^\dagger \Pi H}
\end{split} 
\end{gather}
