\section{Work}
%%%%%%%%%%%%%%%%%%%%%%%%%%%%%%%%%%%%%%%%%%%%%%%%%%%%%%%%%%%%%%%%%
\subsection{The System Model}
%%%%%%%%%%%%%%%%%%%%%%%%%%%%%%%%%%%%%%%%%%%%%%%%%%%%%%%%%%%%%%%%%
Let $x\in\mathbb{R}^{n\times 1}$, $y\in\mathbb{R}^{p\times 1}$ and $w\in\mathbb{R}^{r\times 1}$ be the state, measurement, exogenous vectors, respectively, the SIMO/MIMO LTI system addressed in this paper is stated as, 
\begin{gather} 
    \dot{x}=A x+B_w w+B_u u,\,y=C x+D_w w
\end{gather}
where $A\in\mathbb{R}^{n\times n}$, $B_w\in\mathbb{R}^{n\times r}$, $C\in\mathbb{R}^{p\times n}$ and $D_w\in\mathbb{R}^{p\times r}$. The following additional rank condition
\begin{gather}
    \mathrm{rank}(C)=r<p
\end{gather}
arises in sparse sensor applications\cite{shoukry2015event}, topologies used in Multi Agen Systems(MAS)\cite{olfati2007consensus} and distributed networks \cite{yang2022sensor}, and is also called Strictly Output Redundant(SOR) system \cite{yang2025output}. Here, the system output is overdetermined, therefore, 
\begin{gather}  
    \mathrm{dim}\,\mathcal{N}(C^T)=p-r\geq 1
\end{gather}  
or using orthogonality $C^TC^\perp=0$,
\begin{gather}  
    \mathrm{dim}\,\mathcal{N}(C^\perp)=p-r\geq 1
\end{gather}  
which ensures nontrivial orthogonal component in the output space. 
%%%%%%%%%%%%%%%%%%%%%%%%%%%%%%%%%%%%%%%%%%%%%%%%%%%%%%%%%%%%%%%%%
\subsection{The Observer Models}
%%%%%%%%%%%%%%%%%%%%%%%%%%%%%%%%%%%%%%%%%%%%%%%%%%%%%%%%%%%%%%%%%
The classical Luenberger Observer is defined as,
\begin{gather} 
    \dot{\hat{x}}=A \hat{x}+B_u u+L(y-\hat{y}),\,\hat{y}=C \hat{x}
\end{gather}
where $L\in\mathbb{R}^{n\times p}$ is the observer gain. Assuming $(A,C)$-pair observable, the error dynamics are obtained for the error $e\triangleq x-\hat{x}$ as,
\begin{gather} 
    \dot{e}=(A-LC)e+(B_w-L D_w)w
\end{gather}
The $\mathbb{H}_\infty$ optimal observer for objective function $z=C_ze$ is designed using the following LMI problem\cite{duan2013lmis},
\begin{equation*} \label{eq:p_lmi1}
\begin{split} 
    \min_Y{(\gamma)}\quad \mathrm{s.t.}\\
    \begin{bmatrix}
        (PA-YC)+(PA-YC)^T& PB_w-YD_w& C_z^T\\
        \star& -\gamma I& 0\\
        \star& 0& -\gamma I
    \end{bmatrix}\prec 0\\
    P\succ 0
\end{split} 
\end{equation*}
where the observer gain is recovered with $L=P^{-1}Y$. 

% For constant exogenous inputs the augmented LO is given as,
% \begin{gather} 
%     \begin{bmatrix}\dot{\hat{x}}\\\dot{\hat{w}}\end{bmatrix}=\begin{bmatrix}
%         A& B_w\\0& 0
%     \end{bmatrix}
%     \begin{bmatrix}
%         \hat{x}\\\hat{w}
%     \end{bmatrix}+\begin{bmatrix}
%         L_1\\L_2
%     \end{bmatrix}(y-\hat{y})
%     +\begin{bmatrix}
%         B_u\\0
%     \end{bmatrix}u\\
%     \hat{y}=\begin{bmatrix}
%        C& D_w
%     \end{bmatrix}
%     \begin{bmatrix}
%         \hat{x}\\\hat{w}
%     \end{bmatrix}
% \end{gather}
% with error dynamics,
% \begin{gather}
%     \begin{bmatrix}
%         \dot{e}\\\dot{e}_w
%     \end{bmatrix}
%     =\begin{bmatrix}
%         A-L_1C& B_w-L_1D_w\\
%         -L_2C& -L_2
%     \end{bmatrix}
%     \begin{bmatrix}
%         e\\e_w
%     \end{bmatrix}+\begin{bmatrix}
%         B_u\\0
%     \end{bmatrix}u
% \end{gather}

%%%%%%%%%%%%%%%%%%%%%%%%%%%%%%%%%%%%%%%%%%%%%%%%%%%%%%%%%%%%%%%%%
\subsection{The Projection}
%%%%%%%%%%%%%%%%%%%%%%%%%%%%%%%%%%%%%%%%%%%%%%%%%%%%%%%%%%%%%%%%%
The projection defined using Gauge-Invariance principle is "proposed" as follows,
\begin{gather}
    \Pi\triangleq I-C(C^TC)^{-1}C^T
\end{gather}
where the following properties are satisfied,
\begin{gather}
\begin{split}
    &\Pi C= (I-C(C^TC)^{-1}C^T)C=0\\
    &\Pi C^\perp=(I-C(C^TC)^{-1}C^T)C^\perp=C^\perp,\,C^TC^\perp=0\\
    &\Pi^2=\Pi,\,\Pi\in\mathbb{R}^{p\times p}
\end{split}
\end{gather}
Using the projection on the measured output gives,
\begin{gather}
    \Pi y=\Pi(Cx+D_w w)=\Pi D_w w
\end{gather}
The projection eliminates the state dependency and utilizing
\begin{gather}
    D_w w=C\alpha+C^\perp \beta
\end{gather}
the projection gives,
\begin{gather}
    \Pi y=\Pi C\alpha+\Pi C^\perp \beta=C^\perp \beta
\end{gather}
which is the orthogonal component. 

%%%%%%%%%%%%%%%%%%%%%%%%%%%%%%%%%%%%%%%%%%%%%%%%%%%%%%%%%%%%%%%%%
\subsection{The GI-Luenberger Observer}
%%%%%%%%%%%%%%%%%%%%%%%%%%%%%%%%%%%%%%%%%%%%%%%%%%%%%%%%%%%%%%%%%
The classical Luenberger observer is fed with this orthogonal component forming the GI-Luenberger observer, the observer expression becomes,
\begin{gather} 
    \dot{\hat{x}}=A \hat{x}+B_u u+L(y-\hat{y})+\Lambda \Pi y,\,\hat{y}=C \hat{x}+\Pi y
\end{gather}
where $\Lambda\in\mathbb{R}^{n\times p}$ and the error dynamics are obtained as,
\begin{gather}
    \dot{e}=(A-LC)e+(B_w-LD_w)w-(L-\Lambda)\Pi y
\end{gather}
For 
\begin{gather}
\begin{split}
A=\begin{bmatrix}
0.1495&    0.5832\\
0.8922&    0.8997\\
\end{bmatrix},\,B_w=\begin{bmatrix}
0.3489\\
0.1484
\end{bmatrix}\\
C=\begin{bmatrix}
0.1184&    0.9606\\
0.4265&    0.3471\\
0.7257&    0.3820\\
\end{bmatrix},\,D_w=\begin{bmatrix}
0.8844\\
0.9317\\
0.0529\\
\end{bmatrix}
\end{split}
\end{gather}
the following is obtained,
\begin{gather}
\begin{split}
B_w-LD_w=\\
\begin{bmatrix}
    0.3489 - 0.0529l_{13} - 0.9317l_{12} - 0.8844l_{11}\\
    0.1484 - 0.0529l_{23} - 0.9317l_{22} - 0.8844l_{21}
\end{bmatrix}
\end{split}
\end{gather}
a solution is picked as
\begin{gather}
L=
\begin{bmatrix}
0.3945&0&0\\
0.1678&0&0\\
\end{bmatrix}
\end{gather}

