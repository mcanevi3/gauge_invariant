\subsection{Example 1}
The system is given as, $A=-1$, $B_w=1$,
\begin{equation}
    C=\begin{bmatrix}
        1\\2
    \end{bmatrix}
    ,\,D_w=\begin{bmatrix}
        3\\3
    \end{bmatrix}
\end{equation}
The projection,
\begin{equation}
    \Pi=I-C(C^TC)^{-1}C^T
\end{equation}
is calculated as,
\begin{equation}
    \Pi=\frac{1}{5}\begin{bmatrix}
        4& -2\\-2& 1
    \end{bmatrix}
\end{equation}
Applying the projection on the output,
\begin{equation}
    \Pi y=\begin{bmatrix}
        1.2\\
        -0.6
    \end{bmatrix}w
\end{equation}
is obtained and $w$ is recovered with, $(\Pi D)^\dagger$,
\begin{equation}
    (\Pi D)^\dagger\Pi y=\frac{1}{3}\begin{bmatrix}
        2&  -1
    \end{bmatrix}\begin{bmatrix}
        1.2\\
        -0.6
    \end{bmatrix}w=w
\end{equation}
The observer gain is defined as,
\begin{equation}
    L=\begin{bmatrix}
        l_1& l_2
    \end{bmatrix}
\end{equation}
where the error dynamics are,
\begin{equation}
    \dot{e}=(A-LC)e+(B-LD)w
\end{equation}
Due to the degree of freedom available, 
\begin{equation}
    B-LD=1-3l_1-3l_2=0
\end{equation}
can be enforced and the eigenvalue assignment gives,
\begin{equation}
\begin{split}
    A-LC&=\lambda\\
    -1-l_1-2l_2&=\lambda
\end{split}
\end{equation}
and is combined into,
\begin{equation}
\begin{split}
    -1-l_1-2l_2&=\lambda\\
    -1-l_1-2\left(\frac{1-3l_1}{3}\right)&=\lambda\\
    -3-3l_1-2(1-3l_1)&=3\lambda\\
    -3-3l_1-2+6l_1&=3\lambda\\
    3l_1&=3\lambda+5\\
    l_1&=\frac{3\lambda+5}{3}
\end{split}
\end{equation}
and 
\begin{equation}
    l_2=\frac{-4-3\lambda}{3}
\end{equation}
Hence,
\begin{equation}
    L=\frac{1}{3}\begin{bmatrix}
        3\lambda+5& -4-3\lambda
    \end{bmatrix}
\end{equation}
The states are independent of $w$, therefore the state estimation is not affected by $w$. The output can be corrected using the projection.

%%%%%%%%%%%%%%%%%%%%%%%%%%%%%%%%%%%%%%%%%%%%%%%%%%%%%%%%%%%%%
\subsection{Example 2}


