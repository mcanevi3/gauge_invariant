\documentclass[12pt]{article}

\usepackage[utf8]{inputenc}
\usepackage[turkish]{babel}
\usepackage[T1]{fontenc}

\usepackage{amsmath}
\usepackage{amssymb}
\usepackage{amsthm}
\usepackage{geometry}
\usepackage{graphicx}

\begin{document}
    \shorthandoff{=}
    \section{Ötelemesiz PLL}
    \begin{equation*}
        \begin{split}
            \dot{\theta}_r&=\omega_r,\quad v_r=\sin{(\theta_r)}\\
            \dot{\theta}&=\omega_0+k_v u,\quad v=\sin{(\theta)}\\
            \bar{e}&=v\cdot v_r\\
            e&=\frac{w_c}{s+w_c}\bar{e},\quad \dot{e}=-w_c e+w_c\bar{e}\\
            u&=F(s)e
        \end{split}
    \end{equation*}
    
    \begin{equation*}
        \begin{split}
            \bar{e}&=v\cdot v_r\\
            &=\sin{(\theta)}\sin{(\theta_r)}\\
            &=\frac{1}{2}\left[
                \cos{(\theta_r-\theta)}-\cos{(\theta_r+\theta)}
            \right],\quad \text{LPF}\\
            &\approx \frac{1}{2}\cos{(\theta_r-\theta)}
        \end{split}
    \end{equation*}
    $k_v=2$ olsun,
    \begin{equation*}
        \begin{split}
            \dot{\theta}&=\omega_0+k_v u\\
            \dot{\theta}&=\omega_0+k_v F(s) e\\
            \dot{\theta}&=\omega_0+F(s)\cdot 2 e\\
            \dot{\theta}&\approx\omega_0+F(s)\cdot 2 \frac{1}{2}\cos{(\theta_r-\theta)}\\
            \dot{\theta}&\approx\omega_0+F(s)\cdot \cos{(\theta_r-\theta)}\\
        \end{split}
    \end{equation*}
    
    \begin{figure}
    \centering
        \includegraphics[width=0.95\textwidth]{result1}
    \caption{Sonuçlar-1}\label{fig:result1}
    \end{figure}
    
    \begin{figure}
    \centering
        \includegraphics[width=0.95\textwidth]{model1.png}
    \caption{Simulink modeli}\label{fig:model1}
    \end{figure}
    
    \clearpage
    \section{LMI}
    Sistem denklemleri
    \begin{equation*}
        \begin{split}
            \dot{x}&=Ax+Bu,\quad y=Cx+b+n
        \end{split}
    \end{equation*}
    Gözleyici denklemleri
    \begin{equation*}
        \begin{split}
            \dot{\hat{x}}&=A\hat{x}+Bu+L(y-\hat{y}),\quad y=C\hat{x}+\hat{b}\\
            \dot{\hat{b}}&=-\Gamma (y-\hat{y})
        \end{split}
    \end{equation*}
    Hatalar $e=x-\hat{x}$ ve $e_b=b-\hat{b}$ olmak üzere,
    \begin{equation*}
        \begin{split}
            \dot{e}&=\dot{x}-\dot{\hat{x}}\\
            &=Ax+Bu-A\hat{x}-Bu-L(Cx+b+n-C\hat{x}-\hat{b})\\
            &=Ax-A\hat{x}-LCx-Lb-Ln+LC\hat{x}+L\hat{b}\\
            \dot{e}&=(A-LC)e-Le_b-Ln
        \end{split}
    \end{equation*}
    ve
    \begin{equation*}
        \begin{split}
            \dot{e_b}&=\dot{b-\hat{b}}\\
            &=\dot{b}-\dot{\hat{b}}\\
            &=-\dot{\hat{b}}\\
            &=\Gamma (y-\hat{y})\\
            &=\Gamma (Cx+b+n-C\hat{x}-\hat{b})\\
            &=\Gamma (Ce+e_b+n)\\
            &=\Gamma Ce +\Gamma e_b + \Gamma n
        \end{split}
    \end{equation*}
    Hata dinamikleri,
    \begin{equation*}
        \begin{split}
            \begin{bmatrix}
                \dot{e}\\
                \dot{e_b}\\
            \end{bmatrix}=\begin{bmatrix}
                A-LC& -L\\\Gamma C& \Gamma
            \end{bmatrix}\begin{bmatrix}
                e\\
                e_b\\
            \end{bmatrix}+\begin{bmatrix}
                -L\\
                \Gamma\\
            \end{bmatrix}n
        \end{split}
    \end{equation*}
    Lyapunov fonksiyonu $V=e^TPe+e_b^Te_b/\gamma^2$,
    \begin{equation*}
        \begin{split}
            V&=\begin{bmatrix}
                e\\
                e_b\\
            \end{bmatrix}^T\begin{bmatrix}
                P& 0\\
                0& \gamma^{-2}I
            \end{bmatrix}\begin{bmatrix}
                e\\
                e_b\\
            \end{bmatrix}
        \end{split}
    \end{equation*}
    Türevi,
    \begin{equation*}
        \begin{split}
            \dot{V}&=\begin{bmatrix}
                \dot{e}\\
                \dot{e}_b\\
            \end{bmatrix}^T\begin{bmatrix}
                P& 0\\
                0& \gamma^{-2}I
            \end{bmatrix}\begin{bmatrix}
                e\\
                e_b\\
            \end{bmatrix}+\begin{bmatrix}
                e\\
                e_b\\
            \end{bmatrix}^T\begin{bmatrix}
                P& 0\\
                0& \gamma^{-2}I
            \end{bmatrix}\begin{bmatrix}
                \dot{e}\\
                \dot{e}_b\\
            \end{bmatrix}\\
            &=
            \begin{bmatrix}
                e\\
                e_b\\
            \end{bmatrix}^T
            \begin{bmatrix}
                (A-LC)^T& C^T\Gamma^T\\-L^T& \Gamma^T
            \end{bmatrix}\begin{bmatrix}
                P& 0\\
                0& \gamma^{-2}I
            \end{bmatrix}\begin{bmatrix}
                e\\
                e_b\\
            \end{bmatrix}
            +n^T\begin{bmatrix}
                -L^T&\Gamma^T
            \end{bmatrix}\begin{bmatrix}
                P& 0\\
                0& \gamma^{-2}I
            \end{bmatrix}\begin{bmatrix}
                e\\
                e_b\\
            \end{bmatrix}\\
            &+\begin{bmatrix}
                e\\
                e_b\\
            \end{bmatrix}^T\begin{bmatrix}
                P& 0\\
                0& \gamma^{-2}I
            \end{bmatrix}
            \begin{bmatrix}
                A-LC& -L\\\Gamma C& \Gamma
            \end{bmatrix}\begin{bmatrix}
                e\\
                e_b\\
            \end{bmatrix}+\begin{bmatrix}
                e\\
                e_b\\
            \end{bmatrix}^T\begin{bmatrix}
                P& 0\\
                0& \gamma^{-2}I
            \end{bmatrix}\begin{bmatrix}
                -L\\
                \Gamma\\
            \end{bmatrix}n\\
            &=
            \begin{bmatrix}
                e\\
                e_b\\
                n\\
            \end{bmatrix}^T
            \begin{bmatrix}
            P(A-LC)+(A-LC)^TP& \gamma^{-2}C^T\Gamma^T -PL &-PL\\
            \gamma^{-2}\Gamma C-L^TP& \gamma^{-2}\Gamma+\gamma^{-2}\Gamma^T  &\gamma^{-2}\Gamma\\
            -L^TP& \gamma^{-2}\Gamma^T  &0
            \end{bmatrix}
            \begin{bmatrix}
                e\\
                e_b\\
                n\\
            \end{bmatrix}
        \end{split}
    \end{equation*}
    $\matbb{H}_\infty$ problemi
    \begin{equation*}
        \begin{split}
            \dot{V}+\begin{bmatrix}e& e_b& n\end{bmatrix}^T\begin{bmatrix}e& e_b& n\end{bmatrix}-\gamma_n^2 n^Tn<0
        \end{split}
    \end{equation*}
    ile 
    \begin{equation*}
        \begin{split}
            \begin{bmatrix}
            P(A-LC)+(A-LC)^TP+I& \gamma^{-2}C^T\Gamma^T -PL &-PL\\
            \gamma^{-2}\Gamma C-L^TP& \gamma^{-2}\Gamma+\gamma^{-2}\Gamma^T  &\gamma^{-2}\Gamma\\
            -L^TP& \gamma^{-2}\Gamma^T  &-\gamma_n^2I
            \end{bmatrix}\prec 0
        \end{split}
    \end{equation*}
    kullanılarak optimizasyon problemi,
    \begin{equation*}
        \begin{split}
            &\min{(\rho_n)}\quad \text{s.t.}\\
            &P\succ 0\\
            &\begin{bmatrix}
            A^TP+PA-YC-C^TY^T+I& \rho C^T\Gamma^T-Y &-Y\\
            \rho\Gamma C-Y^T& \rho\Gamma+\rho\Gamma^T  &\rho\Gamma\\
            -Y^T& \rho\Gamma^T  &-\rho_nI
            \end{bmatrix}\prec 0
        \end{split}
    \end{equation*}
    olarak elde edilir ve $Y\triangleq PL$, $\rho_n\triangleq\gamma_n^2$, $\rho\triangleq\gamma^{-2}$
    $\rho=1$ için
    \begin{equation*}
        \begin{split}
            &\min{(\rho_n)}\quad \text{s.t.}\\
            &P\succ 0\\
            &\begin{bmatrix}
            A^TP+PA-YC-C^TY^T+I&  C^T\Gamma^T-Y &-Y\\
            \Gamma C-Y^T& \Gamma+\Gamma^T  &\Gamma\\
            -Y^T& \Gamma^T  &-\rho_nI
            \end{bmatrix}\prec 0
        \end{split}
    \end{equation*}
    olarak elde edilir.
    
\end{document}